%----------------------------------------------------------------------------------------
%	DOCUMENT CONFIGURATIONS
%----------------------------------------------------------------------------------------

\documentclass[11pt]{article}
\usepackage{cite}
\usepackage{url}
\usepackage[utf8]{inputenc}
\usepackage[T1]{fontenc}
\usepackage{fullpage}
\usepackage{mathpazo}
\usepackage{book tabs, multi row, tabularx, tabu, lscape} 			% Table tools
\usepackage[singlelinecheck=off]{caption}
\newcommand{\head}[1]{\textnormal{\textbf{#1}}}
\renewcommand{\arraystretch}{1.1}
\renewcommand{\rmdefault}{bch}
\renewcommand{\sfdefault}{phv}
\usepackage[round,comma,authoryear]{natbib}
\usepackage[protrusion=true,expansion=true]{microtype}				% Better typography
\usepackage{amsmath,amsfonts,amsthm}								% Math packages
\usepackage[pdftex]{graphicx}
%\normalfont % in case the EC fonts aren't available

\title{\textbf{Clustering Trees into Topological Classes \\ Project Plan}} % Title
\author{Kevin Gori} % Author name
\renewcommand{\bibsection}{\Large\textbf{References}\small\vspace{8pt}}

\begin{document}

\maketitle % Insert the title, author and date

\begin{tabular}{l}
Supervisors: Nick Goldman, Christophe Dessimoz % Instructor/supervisor
\end{tabular}

\setlength\parindent{0pt} % Removes all indentation from paragraphs

%\renewcommand{\labelenumi}{\alph{enumi}.} % Make numbering in the enumerate environment by letter rather than number (e.g. section 6)

%--------------------------------------------------------------------------
%	SECTION
%--------------------------------------------------------------------------

\section{Objective}
Phylogenetic inference on modern datasets containing large numbers of 
genes from a group of taxa will in most cases produce discordant trees. 
This project aims to identify and characterise this discordance by applying 
a variety of clustering methods to a matrix of tree distances derived from 
single-gene tree inferences.

%--------------------------------------------------------------------------
%	SECTION
%--------------------------------------------------------------------------

\section{Principles}
%(NB: using 'marker' to represent any sequences, and 'gene' to represent 
%protein coding)
\subsection{Introduction}
Phylogenetic inference from molecular sequences aims to provide a description of the evolutionary history of the marker sequences being analysed. The tree produced summarises two aspects of the evolutionary process: the branching order of the tree corresponds to the order of duplications of the marker sequences, and the branch lengths reflect the process of evolution along the branch -- they estimate the evolutionary rate according to the model, multiplied by the time spent evolving. The tree describes the history of the marker sequences used, but if we are investigating species history we may wish to conflate the marker's branching history with the history of speciation events. However, there are many reasons why a segment of DNA can have a different history to the organism it resides in, including population genetic events such as horizontal gene transfer, recombination, hybridisation and incomplete lineage sorting, and stochastic variation in the substitution process which results in the segment evolving differently by chance. This kind of gene-tree, species-tree incongruence makes single marker phylogenies unreliable for estimating species trees.\\
To get a better estimate of the species history we can use several markers. An increased sample size should allow us to average out some of the stochastic variation, and we can hope that the speciation signal is present in enough of the markers that we can detect it as a common theme. Similarly, markers sharing non-speciation history can also be grouped together. Groups of markers sharing history can be found by clustering them according to their topological similarity, and by determining the number and relative sizes of the clusters found we can characterise the amount of discordance in the dataset. The process generating the discordance is not specified, but some inferences can be made, for example if markers from a particular class are found in a contiguous block in the genome, then we may be able to infer recombination breakpoints in their vicinity.

\subsection{Modelling multiple classes}
Suppose we have a dataset consisting of $M$ alignments on $n$ species. We want to estimate the number of classes, $K$, and for each class $k_{1 \leq i \leq K}$ estimate the underlying tree with $2n-3$ branch lengths. We can model this situation with increasing complexity:

%\begin{landscape}
  \begin{table}[!htbp]
  \caption{General modelling scheme}\label{tab:table}
    \begin{tabu} to \linewidth {X[1.5,l]
                                X[0.5,c]
                                X[1,c]
                                X[1,c]
                                X[3,l]}
      \toprule[1.5pt]
      \head{Model} & \head{Topologies} & \multicolumn{2}{c}{\head{Parameters}} & \head{Notes}\\
        & & Branch lengths & Scale factors ($\tau$) & \\
      \midrule
      Supermatrix 	& $1$ & $2n-3$ &	0 & All markers are generated by the same topology\\ 
      Multiclass Supermatrix & $K$ & $K(2n-3)$ &	 0 & Markers are generated by $K$ class topologies; 
      each has the same underlying branch lengths\\ 
      Proportional Branch Lengths & $K$ & 	$K(2n-3)$ &	$M-K$ & Markers are generated by $K$ class topologies;
      each marker within the class has a scaled set of branch lengths \\ 
      General Branch Lengths & $K$ & $M(2n-3)$	& 0 & Markers are generated by $K$ class topologies;
      each marker has independent branch lengths\\ 
      \bottomrule[1.2pt]
    \end{tabu}
    \end{table}
%\end{landscape}


%--------------------------------------------------------------------------
%	SECTION
%--------------------------------------------------------------------------
\section{Methods}

\subsection{Tree Inference}
I align sequences using some alignment method (for example Prank \citep{Loytynoja:2005cb}), and then infer trees using either a Maximum-Likelihood approach, for which I use PhyML \citep{Guindon:2010gf}, or a distance based approach, TreeCollection, for which distances and variances between sequences in an alignment are estimated using Darwin, with the resulting distance-variance matrices being the input for TreeCollection, which uses a Least-Squares approach to reconstruct the tree.

%--------------------------------------------------------------------------
%	SECTION
%--------------------------------------------------------------------------

\subsection{Estimating Tree Distances}

As a way of avoiding representing trees as vectors of features observed in a defined space I use pairwise distances to construct a distance matrix as a basis for clustering. A number of distance metrics are available for comparing phylogenetic trees. I chose to use the following (not an exhaustive list of all metrics available).
\begin{itemize}
	\item Distance Metrics:
	\begin{itemize}
		\item Robinson-Foulds Distance \citep{Robinson:1981ti}
		\item Euclidean Distance (the Branch Length Distance of \citet{Kuhner:1994ub})
		\item Geodesic Distance \citep{Billera:2001ii}
	\end{itemize}
\end{itemize}
The first two of these measures are calculated using the Python module DendroPy \citep{Sukumaran:2010id}. The fourth is calculated using a modified version of the GeoMeTree program \citep{Kupczok:2008bh}. The unweighted Robinson-Foulds measure is based only on topology, the Branch Length distance is based only on branch lengths and the Geodesic distance is explicitly based on topology and branch lengths.

%--------------------------------------------------------------------------
%	SECTION
%--------------------------------------------------------------------------

\subsection{Clustering Methods}
Clustering is a process of separating data points into groups, the members of which share common characteristics more than they share the characteristics with members of other groups. There is a large number of methods available for doing clustering. 

\begin{enumerate}
	\item Hierarchical Clustering Methods
	\begin{itemize}
		\item Single-linkage
		\item Complete-linkage
		\item Average-linkage
		\item Ward-linkage
	\end{itemize}
	\item Exemplar Methods
	\begin{itemize}
		\item K-medoids (k-centres)
		\item Affinity Propagation
	\end{itemize}
	\item Matrix Decomposition Methods
	\begin{itemize}
		\item Classical Multidimensional Scaling + K-means
		\item Spectral Clustering + K-means
	\end{itemize}
\end{enumerate}

%---------------------

\subsubsection{Hierarchical Clustering}
Possibly the simplest methods which can act directly on distance matrices are (agglomerative) hierarchical clustering. These methods iteratively combine the two closest points, representing them by a single point. How this point is taken to relate to the remaining points is defined by the linkage method used. Let $u$ be the newly defined cluster formed by combining points $s$ and $t$, and $v$ be any other point (or cluster) in the dataset, and let $d(u,v)$ be the distance between $u$ and $v$:
\begin{itemize}
	\item Single-linkage - closest member point 
		\begin{equation*}
		d(u,v)=\min_{i \in u, j \in v} (d(i,j))
		\end{equation*}
\item Complete-linkage - furthest member point 
		\begin{equation*}
		d(u,v)=\max_{i \in u, j \in v} (d(i,j))
		\end{equation*}

	\item Average-linkage - mean of member distances
		\begin{equation*}
		d(u,v)=\frac{1}{\left|{u}\right|\left|{v}\right|}\sum_{i \in u}\sum_{j \in v} d(i,j)
		\end{equation*}

	\item Ward-linkage - minimum variance
		\begin{align*}
		d(u,v)=\sqrt{ \frac{\left|v\right| +  \left|s\right|}{T}d(v,s)^2 + \frac{\left|v\right| +  \left|t\right|}{T}d(v,t)^2 + \frac{\left|v\right|}{T}d(s,t)^2} \\
		T=\left|v\right| +  \left|s\right| + \left|t\right|
		\end{align*}
	
\end{itemize}

%--------------------

\subsubsection{Exemplar Methods}
Other methods acting directly on distance matrices are k-medoids (also known as k-centres), and affinity propagation \citep{Frey:2007hs}. These search for points in the data set to act as exemplars, that is cluster centres to which surrounding points are grouped. K-medoids requires that the number of clusters be set in advance, whereas with affinity propagation the number of clusters depends on the value of a tuning parameter, called the preference, which can take a value between $\min(d)$ and $\max(d)$, and is usually set as the median.

%--------------------

\subsubsection{Matrix Decomposition Methods}
Classical multidimensional scaling maps distances to points in a multidimensional space. The distance matrix is double-centred (from each element subtract the row mean, subtract the column mean, add the overall mean and divide by -2). This eigenvectors of this matrix - the Gower matrix - corresponding to the $k$ largest eigenvalues map the data into a $k$-dimensional space. The points can then be clustered using k-means.\\
Spectral decomposition is carried out using the Ng-Jordan-Weiss (NJW) algorithm \citep{Ng:2002tj} with local scaling approach of \citet{Perona:2004tk}.

%----------------------------------------------------------------------------------------
%	SECTION 
%----------------------------------------------------------------------------------------

\section{Investigating the clustering procedures}

%----------------------------------------------------------------------------------------
%	BIBLIOGRAPHY
%----------------------------------------------------------------------------------------

\newpage
\bibliography{projectplan.bib}
\bibliographystyle{unsrtnat}

\end{document}